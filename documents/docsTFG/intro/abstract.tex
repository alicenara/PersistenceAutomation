\section*{Abstract} 
\addcontentsline{toc}{section}{Abstract}

Cybersecurity is a field that is becoming more important over time, as the number of cyberattacks on all kinds of organizations is growing every year. 
Since the impact of those attacks increases in time (which translates into greater losses to big companies), 
%cybersecurity is becoming a top priority for businesses around the globe. That is why 
it is essential to invest in security equipment, tools, people and/or services in order to be as protected as possible against all kinds of cyber threats.

%Two of the most common cybersecurity services are endpoint and network security evaluations, where cybersecurity professionals test companies' protection solutions against different techniques and tactics used in real life cyberattacks, one of them being \textit{persistence}. Persistence consists of techniques that adversaries use to keep access to systems across restarts, changed credentials, and other interruptions that could cut off their access; and includes the use of \textit{backdoors} in order to be able to reach the system over the Internet.


Two of the most common cybersecurity services are endpoint and network security evaluations, where professionals test a company's antimalware software against different techniques used in real-life attacks, like the ones classified as \textit{persistence}: procedures to re-execute a file or a command, or to reconnect with controlled servers, following reboots or process terminations. 
%. And some of the most common tasks performed during those cyberattacks, both by generic malware and by sophisticated groups, are related to the \textit{persistence} tactic: 

Persistence techniques are used regularly because they are crucial in most intrusions (when an attack has succeeded in accessing internal computers of an enterprise), since losing connection with the compromised machine can make the whole operation fail.

This project collects information about different ways of deploying persistence in diverse operating systems (both Windows and Linux) and services (Active Directory), focusing on the most used in recent attacks. This information is already on the Internet, but it is scattered and sometimes written in overly technical language, making it difficult to understand.  

Additionally, an automation tool is developed to deploy persistence easily and faster on computers. This tool is composed of several scripts, depending on the base operating system, and could be very useful when performing the aforementioned security evaluations.

In short, the final goal of this project is to make lots of resources available that can be used during security assessments, to help identify flaws and thus achieve better protected systems.

\pagebreak
\section*{Resumen} 

La ciberseguridad es un campo que cada vez cobra más importancia, ya que año tras año crece el número de ciberataques a todo tipo de organizaciones. 
Dado que el impacto de dichos ataques es cada vez mayor (lo que se traduce en mayores pérdidas para las grandes empresas), es fundamental invertir en equipos, herramientas, personal y/o servicios de seguridad para estar lo más protegidos posible frente a todo tipo de ciberamenazas.

Dos de los servicios de ciberseguridad más comunes son las evaluaciones de seguridad en los equipos de usuario y en las redes, donde se ponen a prueba los programas antimalware contratados por el cliente, contra diferentes técnicas utilizadas en ataques reales, como las clasificados como \textit{persistencia}: procedimientos para volver a ejecutar un archivo o un comando, o para volver a establecer comunicación con un servidor remoto, después de que el ordenador se haya reiniciado o el proceso haya finalizado.  

Durante las intrusiones, entendiendo "intrusión" como un ataque que ha conseguido acceder a ordenadores de la red interna de alguna empresa, las técnicas de persistencia son cruciales, ya que perder la conexión con el equipo comprometido podría poner en riesgo todo el operativo.

Este proyecto recopila información sobre los diferentes métodos de desplegar persistencia en varios sistemas operativos (tanto Windows como Linux) y en servicios (como el de Directorio Activo), centrándose en las técnicas más utilizadas en los ataques de hoy en día. Esta información se encuentra también en Internet, pero está dispersa y a veces escrita en un lenguaje muy técnico, dificultando su comprensión.

Además, se ha desarrollado una herramienta de automatización para poder realizar el despliegue de persistencia de forma rápida y sencilla. Esta herramienta se compone de varios scripts, adaptados a diferentes sistemas operativos, y puede resultar muy útil cuando se realizan las evaluaciones de seguridad mencionadas anteriormente.

En resumen, el objetivo final de este proyecto es poner a disposición una gran cantidad de recursos que pueden ser utilizados durante las auditorías de seguridad, para ayudar a identificar los fallos del equipo y conseguir así sistemas mejor protegidos.

\pagebreak
\section*{Resum} 

La ciberseguretat és un camp que cada cop té més importància, ja que any rere any creix el nombre de ciberatacs a tota mena d'organitzacions. 
Atès que l'impacte d'aquests atacs és cada vegada més gran (la qual cosa es tradueix en majors pèrdues per a les grans empreses), és fonamental invertir en equips, eines, personal i/o serveis de seguretat per estar el més protegits possible davant de tota mena de ciberamenaces.

Dos dels serveis de ciberseguretat més comuns són les avaluacions de seguretat als equips d'usuari i a les xarxes, on es posen a prova els programes antimalware contractats pel client, davant diferents tècniques utilitzades en atacs reals, com les classificades com a \textit{persistència}: procediments per tornar a executar un fitxer o una comanda, o per tornar a establir la comunicació amb un servidor remot, després de que l'ordinador s'hagi reiniciat o de que el procés hagi finalitzat.

Durant les intrusions, entenent "intrusió" com un atac que ha aconseguit accedir a ordinadors de la xarxa interna d'alguna empresa, les tècniques de persistència són crucials, donat que perdre la connexió amb l'equip compromès podria posar en risc tot l'operatiu.

Aquest projecte recopila informació sobre els diferents mètodes per desplegar persistència tant a diversos sistemes operatius (Windows i Linux) com en serveis (com el de Directori Actiu), centrant-se en les tècniques més usades als atacs d'avui dia. Aquesta informació també es troba a Internet, però està dispersa i de vegades escrita en un llenguatge molt tècnic, fent-ne difícil la seva comprensió.

A més, s'ha desenvolupat una eina d'automatització per poder fer el desplegament de la persistència de forma ràpida i senzilla. Aquesta eina es compon de diversos scripts, adaptats a diferents sistemes operatius, i pot resultar molt útil quan es realitzen les avaluacions de seguretat esmentades anteriorment.

En resum, l'objectiu final d'aquest projecte és posar a disposició una gran quantitat de recursos que poden ser emprats durant les auditories de seguretat, per ajudar a identificar millor els problemes de seguretat dels equips, i aconseguir així sistemes millor protegits.

