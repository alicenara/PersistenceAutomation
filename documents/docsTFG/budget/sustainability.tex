\section{Sustainability report}
After the study of the project carried out in the previous sections, in this part an assessment is made focusing on its sustainability values. Using the proposed "Sustainability Matrix"\cite{susMatrix}, three different dimensions are evaluated in this study: the economic, the social, and the environmental one.
Each area is also divided into three parts: development (PPP), lifetime, and risks.

% being the social area the one with a slightly major importance.

\subsection{Economic sustainability}
When evaluating the economic dimension, not only the resources specified in the budget section apply (section \ref{ssec:costOfResources}) but also the unexpected costs any project could have, even if they have already been covered preventively as a contingency method.

\paragraph{Development}
All costs (human and material resources) that were necessary to build the project are set and explained in sections \ref{ssec:resources} and \ref{ssec:costOfResources}. An additional 10\% was added to the final budget as well, to be used in unforeseen situations.

The final calculated budget is 12.473,45€, which is not a high amount taking in mind that it took 6 months to finish this project.

Doing all this project with less budget and in less time seems a very difficult job to accomplish because a big part of the budget is expended on people researching the information and building the tool. Almost all material resources are free and only some budget can be saved if non-free operating systems (like Windows) are discarded, but considering that most of the persistence techniques are deployed on Windows, it would make the project less useful.

\paragraph{Lifetime}
Additional costs for maintenance and updates after the project finishes are difficult to predict as it is Open Source, which means that while future developers and contributors will provide both code (that has a cost), and use equipment (hardware and software resources), at the same time no one will expect to be paid (unless this project is continued by members of a corporation, which is unlikely).   

\pagebreak
\paragraph{Risks}
\begin{itemize}
\item Even though it is expected to be useful in the years to come, software is always evolving: new applications are created every day, and old ones receive updates or people stop using them. This project may become obsolete sooner than expected if new technologies arise unexpectedly, like quantum computing.
\end{itemize}

Finally, this project's economical viability grade is a 9 as there are only a few things that can be changed to get a lower price, but overall it is affordable.

\subsection{Social sustainability}
The social dimension evaluates the impact that this project can have both on the developers and all other people that it could reach.

\paragraph{Development}
All the people involved in the project, described in section \ref{ssec:costOfResources}, learned a lot about this topic, that was unknown to some. Even the ones with previous knowledge on the topic discovered new strategies to deploy persistence, as most of them are not used frequently during security assessments.

And with the acquired knowledge, people are able to better protect their systems, which is the end goal of this project.

\paragraph{Lifetime}
This project is intended for educational and professional purposes, so it could be beneficial to future researchers and students. Also, as some techniques have been in use for more than 10 years, it is expected that the gathered information would be of help at least for 5 more years.

But it is common for malicious cyber threat actors to use or adapt Open Source tools to commit crimes. So, even though this tool is not very sophisticated and does not use never seen techniques, it is still possible to be employed for the wrong motivation.

The impact may be moderate either if it is used by criminals or by cybersecurity professionals due to the automation being on basic levels. 
%After all its goal is to improve at least a little the analysts' work. 

\pagebreak
It is important to note that nowadays, software often gets updated when a flaw in its code or design is being abused by criminals. There are several examples of tools created by researchers and security analysts that were built to do their work better or faster, and that eventually end up being used by criminals. Is in that situation that companies put more effort into preventing the abuse of their functionalities. That is why new tools are always welcome, even if used wrongly.

\paragraph{Risks}
\begin{itemize}
\item The research could be difficult to find through the Internet, and the tool may not be suited for all environments, which can lead to nobody using it, and finally the discontinuation of the project. 
\item It may be used by cybercriminals, and consequently be a problem for users, although it would have a limited impact as described before.
\end{itemize}

This part will have a score of 7 because on the one hand it is an Open Source project and is meant to help people, but on the other hand, it can cause problems if the tool falls into the wrong hands. 


\subsection{Environmental sustainability}

Many factors influence the environmental dimension, since not only the emissions generated on the development are taken into account, but also all the actions that can be done to reduce the carbon footprint during all this project's phases.

\paragraph{Development}
All resources used in the project have been precised in sections \ref{sec:budget} - \textit{Budget estimation} and \ref{sec:sched} - \textit{Scheduling}.

The only physical resource needed to build this project was a computer, which was reused because this project did not require a high-performance system or any other specific characteristics. Consequently, it did not have a major impact on the environment.

Also, the computer was always on battery-saving mode, which reduces the energy consumed. An estimation of the energy that was used while developing the project is 0,065kWh * \the\value{totalFinalHours} h = 29,12kW. Assuming that each kW generates 0.166kg of C0$_2$, 29,12kW is equivalent to 4,85kg of CO$_2$. 
%, so, in perspective, it is not a lot per day.

It is important to mention that the code was periodically uploaded to an external website (Github), which is composed of servers that have a carbon footprint as well. This footprint cannot be calculated, but that servers also host thousands of other projects, and are more efficient than having a local server to do the version control of the code, which is necessary.

\paragraph{Lifetime}
After the project is finished, the amount of energy that this project will consume cannot be calculated because it depends on multiple factors:
\begin{itemize}
\item If it is used, it will require a minimal electric consumption on the computer.
\item If its development is continued, the computer used in the development will need some electricity, but it depends on the time spent on it.
\item The servers where the code is uploaded will indeed use electricity to allow the code to be available, as stated in the "Development" paragraph, but that website is not hosting only this project, so the server used there is shared with other projects as well, and it is an energy that is going to be consumed regardless of the project being there hosted or not. 
\end{itemize}

Therefore, even though there will be some electrical consumption, it will be a little amount compared to the energy used to develop it.

\paragraph{Risks}
\begin{itemize}
\item The principal risk was that the project would be delayed for various reasons, which would generate more energy consumption and therefore, more CO$_2$.
\end{itemize}

For all the reasons stated above, this project's environmental sustainability value is a 7 because it cannot consume less CO$_2$ as it could not be developed or used without a computer, which causes the calculated emissions. 
