\subsection{Budget control}
\label{ssec:budgetControl}
Even though all the expenses have been taken into account, some deviations on the budget were still possible due to different reasons: hardware failure, new software necessities, etc.

All hardware and software resources have a useful lifetime of 2 years or more, so even if more time was needed to finish the project, the final budget related to those resources would not be increased.

No delays were produced on tasks since various measures were taken to prevent them,  like that a little extra time was added to each task after calculating the strictly necessary one, or that there were reviews after each task was completed, to correct any delay sooner than later.

Additionally, the final budget was incremented an additional 10\% as a contingency measure, meant to be used if needed. So, in the event of requiring more time to finish the tasks or having to pay for unforeseen software, there was a big chance that the final budget would not be modified anyway.

Lastly, the next equations were calculated to determine the possible deviations:
\begin{itemize}
\item Deviation in the developing of a task in costs = (estimated costs - real cost) * final hours
\item Deviation in the developing of a task in time = (estimated time - expended time) * final cost
\item Deviation of a resource in costs = (estimated costs - real cost) * final hours
\item Deviation of a resource in time = (estimated time - expended time) * final cost
\end{itemize}