\newcounter{tTresHours}
\setcounter{tTresHours}{150} 
\subsubsection{T3 - Research and documentation}
\label{sssec:researchTask}

\textbf{Summary}
\begin{table}[ht]
\centering
  \begin{tabular}{| c | c | c |}
  \hline \textbf{Hours assigned} & \textbf{Hours per day} & \textbf{Weeks} \\ \hline  
   \the\value{tTresHours} & 3 & 10    \\ \hline
  \end{tabular}
  \caption{Research and documentation hours review} \vspace{3pt}
  \label{tab:design}
\end{table}

\par{\textbf{Explanation}}

One of the main goals of this project (section \ref{ssec:objectives}) is to research and collect lots of different techniques to implement persistence, as it helps cybersecurity professionals when performing computer audits.\\
For this reason, in this part of the project, the necessary search and documentation were carried out to understand in depth a large number of ways to apply persistence.   

This section was divided into different parts:
\begin{itemize}
\item Persistence history
\item Discovery techniques
\item Backdoors and protocols
\item Persistence techniques on different operating systems
\item General techniques
\item Active Directory persistence

\end{itemize}

It is also worth mentioning that the research performed in this task proved to be very useful when designing and building the tool, as it was used to define which procedures were implemented.

The amount of time spent doing this part was \the\value{tTresHours} hours, spending 3h per day, as this task was done simultaneously with task T2, section \ref{sssec:projectManagementTask}. It was finished in ten weeks, which is 2 months and a half. 

