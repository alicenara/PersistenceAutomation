%\setcounter{tQuatreQuatreHours}{37} 51
\textbf{T4S3: Subtask 3 – Script in Python for Linux}
\label{sssec:scriptPythonTask}

\textbf{Summary}
\begin{table}[!htb]
\centering
  \begin{tabular}{| c | c | c | c |}
  \hline \textbf{Phase} & \textbf{Hours assigned} & \textbf{Hours per day} & \textbf{Weeks} \\ \hline  
  Development & 40 & 4 & 2       \\ \hline
  Testing & 7 & 3 & 3 days       \\ \hline
  Documentation & 4 & 1 & 4 days       \\ \hline \hline
  \textbf{Total} & \the\value{tQuatreQuatreHours} &  & 3  \\ \hline
  \end{tabular}
  \caption{Development - Python script hours review} \vspace{3pt}
  \label{tab:sprint4}
\end{table}

\textbf{Explanation}\\
This subtask's main goal was to build a script in Python both to collect environmental information (discovery techniques) and to deploy persistence using different mechanisms.

This script followed the design created in section T4S2 and was fully tested and commented, in order to end up with the finished script for Linux.

%The techniques implemented on this part are listed in section \ref{ssec:development-linux}
The techniques implemented on this part were:
\begin{itemize}
\item Crontab 
\item \texttt{bashrc} and \texttt{init} persistence
\item Systemd 
\item New users
\item SSH related backdoors
\item \texttt{netcat} reverse shell
\item External backdoor deployment
\end{itemize}

This part took three weeks and a third of the time was spent on the testing and documenting processes, as they were crucial to have this task finished.