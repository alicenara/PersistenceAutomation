\newcounter{tDosHours}
\setcounter{tDosHours}{80} 
\subsubsection{T2 - Project management}
\label{sssec:projectManagementTask}

\textbf{Summary}
\begin{table}[ht]
\centering
  \begin{tabular}{| c | c | c |}
  \hline \textbf{Hours assigned} & \textbf{Hours per day} & \textbf{Weeks} \\ \hline  
   \the\value{tDosHours} & 4 & 4        \\ \hline
  \end{tabular}
  \caption{Project management hours review} \vspace{3pt}
  \label{tab:projectManagement}
\end{table}

\par{\textbf{Explanation}}

%This task corresponds to the GEP part
The basic project and documentation structure was defined in this task, and some research was done. Once finished, an extensive report was obtained which is included in this final documentation, delivered at the end of the project. 

The amount of time spent was about \the\value{tDosHours} hours, working 4 hours per day, finishing it in a month. 

This task had the following parts:
\begin{itemize}
\item Introduction, scope of the project and contextualization
\item State of the art and methodology
\item Time scheduling
\item Economic management and sustainability
\end{itemize}

It provided a solid start because it clearly defined how the project would be developed: deadlines, milestones, needed resources, etc. And that is why all other tasks (excluding the first one, "\textit{Viability of the project}", \ref{sssec:viabilityTask}) were defined and scheduled in this one.

As it can be observed in the Gantt chart on section \ref{ssec:gantt}, this task was partially performed at the same time as the next task, section \ref{sssec:researchTask} (\textit{T3 - Research and documentation}), because the \textit{Research} task delves into some of the concepts and tools that are also documented in this task.